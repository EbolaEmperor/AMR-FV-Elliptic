\documentclass[cn, bibend=bibtex]{elegantpaper}

\title{二维并行AMR程序设计文档}
\author{Wenchong Huang}
\date{\today}
\setcounter{tocdepth}{3}
\everymath{\displaystyle}
\usepackage{esint}
\theoremstyle{plain}
\usepackage{miniplot}
\newtheorem{exercise}{练习}[section]
\newenvironment{solution}{\begin{proof}[\textbf{\emph 解}]}{\end{proof}}

\usepackage{bm}

\begin{document}
\maketitle
\vspace{-1em}
\tableofcontents
\newpage


% \newpage
\section{进程数据交换}
\subsection{class LinearizationHelper}
\begin{itemize}
    \item 功能:提供将不同数据转化成连续内存的辅助函数。
    \item 成员函数:
    \begin{enumerate}
        \item \lstinline|inline void linearIn(const T &input, std::vector<char> &buf)|
        \item[] 模板:\lstinline|template<class T>|
        \item[] public 成员函数, 将\lstinline|input|中数据复制到连续内存\lstinline{buf}中.
        \item \lstinline|inline void linearIn(const std::vector<T> &input, std::vector<char> &buf)|
        \item[] 模板:\lstinline|template <class T>|
        \item[] public 成员函数, 将\lstinline|input|中的数据复制到连续内存\lstinline{buf}中.
        \item \lstinline|inline void linearOut(const std::vector<char> &buf, unsigned &end_pos, T &res)|
        \item[] 模板:\lstinline|template <class T>|
        \item[] public 成员函数, 将\lstinline|buf|中的数据恢复到对应位置\lstinline|res|中.
        \item \lstinline|inline void linearOut(const std::vector<char> &buf, unsigned &end_pos, std::vector<T> &res)|
        \item[] 模板:\lstinline|template <class T>|
        \item[] public 成员函数, 将\lstinline|buf|中的数据恢复到对应位置\lstinline|res|中.
    \end{enumerate}
\end{itemize}
\subsection{class ParallelDataBase}
\begin{itemize}
    \item 功能:并行数据类型的基类.
    \item 模板:\lstinline{template <bool EnableRepeatComm, bool KnowSendInfo, bool KnowRecvInfo>}
    \item 成员变量:
    \begin{enumerate}
        \item \lstinline{Communicatior comm_;}
        % \item \lstinline|bool lock_;|
        \item \lstinline|map<unsigned, size_t> data_length_to_send_;|
        \item[] 保存要发送信息的进程编号和信息长度.
        \item \lstinline|map<unsigned, size_t> data_length_to_recv_;|
        \item[] 保存要接受信息的进程编号和信息长度
        \item\lstinline{map<unsigned, vector<char>> send_buf_;}
        \item[]\lstinline{map<unsigned, vector<char>> recv_buf_;}
        \item[]...
      %   \item \lstinline|enum class InitOpt{kInitDir, kInitFromSend, kInitFromRecv, kStopInit};|
      %   \item \begin{lstlisting}
      % struct SyncOpt{
      %   bool always_init_sync_pair_;
      %   bool always_init_data_length_;
      %   InitOpt init_sync_pair_opt_;
      %   InitOpt init_data_length_opt_;
      % };
      % \end{lstlisting}
      %   % \item \lstinline|map<unsigned, vector<char>> send_buf_;|
      %   % \item 发送的连续内存
      %   % \item \lstinline|map<unsigned, vector<char>> recv_buf_;|
      %   % \item 接收的连续内存
      %   \item \lstinline|Communicator comm_;|
      %   \item[] 通讯器编号.
      %   \item \lstinline|unsigned max_sending_capacity_|
      %   \item[] 允许单次发送的最大长度.
    \end{enumerate}
    \item 成员函数:
    \begin{enumerate}
        \item \lstinline{ParallelDataBase(Communicator comm)}
        \item \lstinline{virtual void communicationBegin() = 0;}
        \item[] 开始通讯.
        \item \lstinline{virtual void communicationEnd() = 0;}
        \item[] 结束通讯.
        % \item \lstinline{Copier(std::vector<unsigned> &exchange_info, bool& is_init_from_recv);}
        % \item[] 从接收端或发送端同步进行信息传递的进程组
        % \item \lstinline{Copier(std::vector<unsigned> &send_info, std::vector<unsigned> &recv_info);}
        % \item[] 直接设置同步信息组.
        % \item \lstinline{void resetInitOpt(const SyncOpt& sync_opt);}
        % \item[] 重设信息同步设置.
        % \item \lstinline{void setDataLength(const TF&handler, const T&data);}
        % \item[] 模板: \lstinline{template <class TF, class T>}
        % \item[] 初始化\lstinline{data_length_to_send_}和\lstinline{data_length_to_recv_}.
        % \item \lstinline{void linearIn(const TF& handler, const T& data);}
        % \item[] 模板: \lstinline{template <class TF, class T>}
        % \item[] 通过\lstinline{handler}将要同步的数据打包到\lstinline{sendBuf_}中.
        % \item\lstinline|void linearOut(const TF& handler, T& data)|
        % \item[] 模板: \lstinline{template <class TF, class T>}
        % \item[] 通过\lstinline{handler}将同步的数据放回\lstinline{data}中.
        % \item\lstinline{void exchangeData()}
        % \item 发送\lstinline{send_buf_}并接收\lstinline{recv_buf_};
        % \item \lstinline|Copier(const map<unsigned, unsigned> &send_info, const map<unsigned, unsigned> &recv_info, const Communicator comm)|
        % \item[] 同时有发送和接受信息的构造函数
        % \item \lstinline|Copier(const map<unsigned, unsigned> &exchange_info, bool init_by_recv, const Communicator comm)|
        % \item[] 发送和接收信息只知道其一的构造函数
        % \item \lstinline|void syncExchangeInfo()|
        % \item[] 同步得到未知的发送或接收信息.
        % \item \lstinline|void exchangeData()|
        % \item[] 进行数据交换,并将数据填入对应位置.
        % \item \lstinline|virtual void rebuildData() = 0|
        % \item[] protected 成员函数.
        % \item[] 将接收的数据重构回对应位置.
    \end{enumerate}
\end{itemize}
\section{网格分割与数据存储}


\subsection{class BaseIterator}

\begin{itemize}
  \item 功能:迭代器基类。
  \item 模板:\lstinline|template<typename T>|

  T 表示迭代器作用的类名,T 必须有成员函数 \verb|.size()|
  \item 成员变量:
  \begin{enumerate}
    \item \lstinline|T& object_|

    protected 成员变量:当前迭代器正在遍历的对象
  
    \item \lstinline|unsigned current_|
    
    protected 成员变量:当前迭代器指向对象中的第几个内容
  \end{enumerate}
  \item 构造函数:
  \begin{enumerate}
    \item \lstinline|BaseIterator(T& obj): object_(obj), current_(0) {}|
  \end{enumerate}
  \item 成员函数:
  \begin{enumerate}
    \item \lstinline|unsigned index() const|

    public 成员函数,返回当前迭代器的索引

    \item \lstinline|virtual bool ok() const {return current_ < object_.size();}|

    public 成员函数,返回能否继续遍历

    \item \lstinline|virtual BaseIterator<T>& operator++ ()|

    public 成员函数,遍历下一个内容,默认实现是 \verb|++current_|,也可以由派生类定义
  \end{enumerate}
\end{itemize}

\subsection{class DisjointBoxLayout}

\begin{itemize}
  \item 功能:存储一些不相交的 Box 布局。
  \item 模板:\lstinline|template<int Dim>|
  \item 成员变量:
  \begin{enumerate}
    \item \lstinline|Vector<Box<Dim>> boxes_|
    
    protected 成员变量:用于存储 Box,按每个 Box 的左下角坐标排序。

    \item \lstinline|Vector<int> procs_|

    protected 成员变量:存储每个 Box 属于哪个进程
  \end{enumerate}
  \item 迭代器: \lstinline|class Iterator : public BaseIterator<const Vector<Box<Dim>>>|
  \begin{enumerate}
    \item \lstinline|const Box<Dim>& operator* () const|
    
    迭代器的 public 成员:返回当前迭代器指向的 Box
  \end{enumerate}
  \item 构造函数:
  \begin{enumerate}
    \item \lstinline|DisjointBoxLayout(const Vector<Box<Dim>>& boxes, const Vector<int>& procs_)|
  \end{enumerate}
  \item 成员函数:
  \begin{enumerate}
    \item \lstinline|unsigned size() const|

    public 成员函数,返回 Box 的数量。

    \item \lstinline|const Box<Dim>& getBox(unsigned i) const|

    public 成员函数,返回第 i 个 Box.

    \item \lstinline|int whichBox(const iVec &idx) const|

    public 成员函数,寻找指标 \lstinline|idx| 在哪个 Box 中,找不到则返回 $-1$。

    \item \lstinline|void checkDisjoint() const|

    protected 成员函数,检查 Box 是否互不相交,由构造函数调用

    \item \lstinline|DisjointBoxLayout<Dim> getRefined(int refRatio=2) const|

    public 成员函数,返回当前 DisjointBoxLayout 的加细版本

    \item \lstinline|DisjointBoxLayout<Dim> getCoarsened(int refRatio=2) const|

    public 成员函数,返回当前 DisjointBoxLayout 的加粗版本

    \item \lstinline|bool operator== (const DisjointBoxLayout<Dim>& rhs) const|

    public 成员函数,判断两个 DisjointBoxLayout 是否相等

    \item \lstinline|Iterator begin() const|

    public 成员函数,获取指向第一个 Box 的迭代器。
    \item \lstinline|unsigned getProcID(unsigned i) const|

    public 成员函数,返回第 \lstinline|i| 个 \lstinline|Box|对应的进程号
    \item \lstinline|unsigned getDataID(unsigned box_id) const|

    public 成员函数,返回第 \lstinline|box_id| 个 Box 对应在进程中的位置。
    \item \lstinline|unsigned numBoxes(unsigned proc_id) const|

    public 成员函数,返回进程 \lstinline|proc_id| 对应的 Box 的数量。
  \end{enumerate}
\end{itemize}


\subsection{class ProblemDomain}

\begin{itemize}
  \item 功能:存储由一系列不相交 Box 构成的物理区域。
  \item 模板:\lstinline|template<int Dim>|
  \item 成员变量:
  \begin{enumerate}
    \item \lstinline|DisjointBoxLayout<Dim> layout_|

    protected 成员变量:物理区域的布局。其中 Box 的加密程度和同层网格保持一致。
  
    \item \lstinline|Vec<Real,Dim> dx_|
    
    protected 成员变量:网格间距

    \item \lstinline|Vec<Real,Dim> x0_|

    protected 成员变量:网格左下角的坐标

    \item \lstinline|int nGhost_|

    protected 成员变量:ghost 区域宽度
  \end{enumerate}
  \item 构造函数:
  \begin{enumerate}
    \item \lstinline|ProblemDomain(const vector<Box<Dim>>& layout, rVec dx, int nGhost=2)|
  \end{enumerate}
  \item 成员函数:
  \begin{enumerate}
    \item \lstinline|const DisjointBoxLayout<Dim>& getLayout() const|

    public 成员函数,返回表示物理区域的 DisjointBoxLayout。
    
    \item \lstinline|const Vec<Real,Dim> getDx() const|

    public 成员函数,返回网格间距。

    \item \lstinline|const Vec<Real,Dim> getX0() const|

    public 成员函数,返回网格左下角坐标。

    \item \lstinline|int getNumGhosts() const|

    public 成员函数,返回 ghost 区域宽度。

    \item \lstinline|ProblemDomain<Dim> getRefined(int refRatio=2) const|

    public 成员函数,返回当前 ProblemDomain 加细版本

    \item \lstinline|ProblemDomain<Dim> getCoarsened(int refRatio=2) const|

    public 成员函数,返回当前 ProblemDomain 加粗版本
  \end{enumerate}
\end{itemize}

\subsection{class LevelGhostBoxes}

\begin{itemize}
  \item 功能:存储 AMR 某层网格所有 Ghost Box 的信息。
  \item 模板:\lstinline|template<int Dim>|
  \item 成员变量:
  \begin{enumerate}
    \item \lstinline|const ProblemDomain<Dim>& domain_|

    private,表示物理区域
    \item \lstinline|const DisjointBoxLayout<Dim> &mesh_|

    private,表示当前层级上的计算网格

    \item \lstinline|int nGhost_|

    private,表示 ghost 区域宽度
    \item \lstinline|Vector<Box<Dim>> ghostedBoxes_|

    private,表示当前层级上的所有 ghost box 区域
    \item \lstinline|Vector<int> belongs_|

    private,\verb|belongs[i]| 表示第 i 个 ghost box 属于哪个 box
    \item \lstinline|Vector<std::optional<int>> location_|

    private,\verb|location[i]| 表示第 i 个 ghost box 的实际位置在哪个 box 内,如果落在粗网格中或者落在物理区域外面,则为空

    \item \lstinline|Vector<std::optional<int>> bdryOfWhichDomain_|

    private,表示每个 ghost box 是第几个 domain 的物理边界,如果不是物理边界,则为空

    \item \lstinline|Vector<std::optional<int>> bdryOfWhichFace_|

    private,表示每个 ghost box 是物理边界的哪个面(左右下上),如果不是物理边界,则为空
  \end{enumerate}
  \item 迭代器:\lstinline|using DisjointBoxLayout<Dim>::Iterator|
  \item 构造函数:
  \begin{enumerate}
    \item \lstinline|LevelGhostBoxes(const ProblemDomain<Dim>& domain,|
    
          \lstinline|                   const DisjointBoxLayout<Dim>& mesh)|
  \end{enumerate}
  \item 成员函数:
  \begin{enumerate}
    \item \lstinline|unsigned size() const|

    public 成员函数,返回存储的 ghost box 数量。

    \item \lstinline|Iterator begin() const|

    public 成员函数,返回指向第一个 ghost box 的迭代器。

    \item \lstinline|int getBelongs(const Iterator& it) const|

    \item \lstinline|std::optional<int> getLocation(const Iterator &it) const|

    \item \lstinline|std::optional<int> getBdryOfWhichDomain(const Iterator &it) const|

    \item \lstinline|std::optional<int> getBdryOfWhichFace(const Iterator &it) const|

    \item \lstinline|void initLevelGhostBoxes()|

    protected 成员函数,根据网格和定义域来初始化所有界面信息,由构造函数调用。
  \end{enumerate}
\end{itemize}


\subsection{class AMRMeshRefine}

\begin{itemize}
  \item 功能:对 AMR 网格做局部加密操作。
  \item 模板:\lstinline|template<int Dim>|
  \item 成员变量:
  \begin{enumerate}
    \item \lstinline|const DisjointBoxLayout<Dim>& mesh_|

    protected 成员变量:存储待加细或加粗的一层 AMR 网格。
  \end{enumerate}
  \item 构造函数:
  \begin{enumerate}
    \item \lstinline|AMRMeshRefine(const DisjointBoxLayout<Dim>& box_layout)|
  \end{enumerate}
  \item 成员函数:
  \begin{enumerate}
    \item \lstinline|DisjointBoxLayout<Dim> makeLayout(const LevelData<bool, Dim>& tags, int refRatio=2) const|

    public 成员函数,根据用户传入的 tag,返回局部加密后的矩形区域,
    下面的函数是实现该功能所需要的。

    \item \lstinline|virtual void makeBoxesParallel(const LevelData<bool, Dim>& tags, std::vector<Box<Dim>>& refBoxes) const = 0|
    
    protected 成员函数,根据用户传入的 tag,得到 \textbf{当前进程中} 需要加密的矩形 refBoxes。要设计一种算法把零散的标记处理成整块的矩形们。具体算法由派生类实现。

    \item \lstinline|void sendBoxesParallel(const std::vector<Box<Dim>>& refBoxes) const|

    protected 成员函数,将当前进程的加密 Box 发送给其它进程,在 getRefineBoxesThisProc 之后调用。

    \item \lstinline|void receiveBoxesParallel(std::vector<Box<Dim>>& refBoxes, std::vector<int>& procs) const|

    protected 成员函数,接受其它进程的加密 Box,用 procs 数组记录接受的 Box 来自哪个进程,在 sendBoxesParallel 之后调用。

    % \item \lstinline|DisjointBoxLayout<Dim> localCoarsen(const LevelData<bool, Dim, 1>& tags, const unsigned& refRatio=2)|

    % public 成员函数,根据用户传入的 tag,返回局部加粗后的的矩形区域
    % \begin{note}
    %     由于进度紧张,localCoarsen 函数暂时不实现,之后做时间积分时再来实现。
    % \end{note}
  \end{enumerate}
\end{itemize}

\subsection{class BRMeshRefine : public AMRMeshRefine}

\begin{itemize}
  \item 功能:实现 BR 局部加密算法。
  \item 模板:\lstinline|template<int Dim>|
  \item 成员变量:如果实现算法需要,可以加
  \item 成员函数:
  \begin{enumerate}

    \item \lstinline |void makeBoxesParallel(const LevelData<bool, Dim>& tags, std::vector<Box<Dim>>& refBoxes, int refRatio) const|
    
    protected 成员函数,根据用户传入的 tag,得到当前进程中需要加密的矩形区域 refBoxes。采用 BR 算法把零散的标记处理成整块的加密区域。
  \end{enumerate}
\end{itemize}

% \subsection{class LoadBalancer}

% \begin{itemize}
%   \item 功能:负载均衡器。
%   \item 模板:\lstinline|template<int Dim>|
%   \item 成员变量:
%   \begin{enumerate}
%     \item \lstinline|const AMRMeshHierarchy<Dim> &oldHandler_|
%   \end{enumerate}
%   \item 构造函数:
%   \begin{enumerate}
%     \item \lstinline|LoadBalancer(const AMRMeshHierarchy<Dim> &handler)|
%   \end{enumerate}
%   \item 成员函数:
%   \begin{enumerate}
%     \item \lstinline|AMRMeshHierarchy<Dim> getBalanced() const|

%     public 成员函数,返回负载均衡后的网格。
%   \end{enumerate}
% \end{itemize}


% \newpage
% \section{定义域表示}

\subsection{class LevelData}

\begin{itemize}
  \item 功能:用于存储 AMR 某一层网格中一个进程上所需的数据,必须建立于 \lstinline|DisjointBoxLayout|。同时负责进程间数据传输。
  \item 模板:\lstinline|template<typename T, int Dim>|

    \lstinline|T| 表示基本数据类型,\lstinline|Dim| 表示空间的维度
  \item 成员变量:
  \begin{enumerate}
    \item \lstinline|const DisjointBoxLayout<Dim> mesh_|

    private 成员变量,表示当前 LevelData 对象所属的 \lstinline|DisjointBoxLayout|。

    \item \lstinline|Vector<int> index_|

    private 成员变量,表示存储数据的 Box 对应 \lstinline|layout_| 中 Box 的下标。
    
    \item \lstinline|unsigned nComps_|

    private 成员变量,表示当前 LevelData 对象具有几个分量
    
    \item \lstinline|Vector<int> centering_|

    private 成员变量,表示每一个分量的 centering 类型

    \item \lstinline|int nGhost_|

    private 成员变量,表示每个 Box 周围的 Ghost 层数

    \item \lstinline|Vector<Vector<Tensor<T,Dim>>> data_|

    private 成员变量,表示存储的数据。
    第一维下标是 Box,第二维下标是 component。
  \end{enumerate}
  \item 构造函数:
  \begin{enumerate}
    \item \lstinline|LevelData(const DisjointBoxLayout<Dim>& mesh, |

          \lstinline|            const Vector<int>& centering,|

          \lstinline|            unsigned nComps = 1)|
         
          \lstinline|            unsigned nGhost = 0)|
  \end{enumerate}
  \item 迭代器:\lstinline|class Iterator : public BaseIterator<LevelData<T,Dim>>|

  或 const 限定版本: \lstinline|class ConstIterator|
  \begin{enumerate}
    \item \lstinline|unsigned box_id|

    迭代器的 private 成员变量,表示当前迭代器所指向的块状数据对应的 Box 编号
    \item \lstinline|BaseIterator<LevelData<T,Dim>>& operator++ ()|

    迭代器的 public 成员函数,让当前迭代器指向下一个由当前进程负责的块状数据
    \item \lstinline|Vector<Tensor<T,Dim>>& getData()|

    迭代器的 public 成员函数,返回当前迭代器指向的块状数据
    \item \lstinline|Box<Dim> getValidBox(unsigned comp = 0) const|
    
    迭代器的 public 成员函数,返回当前迭代器指向的块状数据的第 comp 个分量所在的 Box,
    该 Box 和数据的 centering 一致,且不包含 Ghost 区域。

    \item \lstinline|Box<Dim> getGhostedBox(unsigned comp = 0) const|
    
    迭代器的 public 成员函数,同上,但包含 Ghost 区域。
  \end{enumerate} 
  \item 成员函数:
  \begin{enumerate}
    \item \lstinline|Iterator begin()|

    public 成员函数:返回指向当前进程负责的第一个块状数据的迭代器

    \item \lstinline|ConstIterator const_begin() const|

    public 成员函数:返回指向当前进程负责的第一个块状数据的 const 限定版本迭代器
    \item \lstinline|const DisjointBoxLayout<Dim>& getMesh() const|

    public 成员函数:返回数据所在的网格。

    \item \lstinline|unsigned size() const|

    public 成员函数:返回该 LevelData 中存了几个 Box 的数据。

    \item \lstinline|unsigned getnComps() const|

    public 成员函数:返回该 LevelData 具有几个分量。

    \item \lstinline|int getCentering(unsigned comp = 0) const|

    public 成员函数:返回第 \lstinline|comp| 个分量的 centering 类型。

    \item \lstinline|std::array<Tensor<T,Dim>,nComps>& getBoxData() (unsigned box_id)|

    public 成员函数:返回编号为 \lstinline|box_id| 的 Box 对应的数据(另有 const 版本)。

    \item \lstinline|void exchangeAll()|

    public 成员函数:和其它进程中的同一个 LevelData 做数据交换,获取数据并填充在内部 Ghost 中.
  \end{enumerate} 
\end{itemize}

\subsection{class AMRMeshHierarchy}

\begin{itemize}
  \item 功能:AMR 网格管理器,管理所有 AMR 网格层的信息。
  \item 模板:\lstinline|template<int Dim>|
  \item 成员变量:
  \begin{enumerate}
    \item \lstinline|std::vector<ProblemDomain<Dim>> &domains_|

    每一层网格对应的 ProblemDomain
    \item \lstinline|std::vector<DisjointBoxLayout<Dim>> &meshes_|

    每一层 AMR 网格
    \item \lstinline|std::vector<LevelGhostBoxes<Dim>> &ghostBoxes_|

    每一层 AMR 网格中的 ghost box 信息
    \item \lstinline|std::vector<unsigned> refRatios_|

    每一层网格的加密率

    \item \lstinline|std::vector<std::vector<unsigned>> parents_|

    \lstinline|parents_[i][j]| 表示第 $i$ 层网格的第 $j$ 个 Box 所覆盖的粗 Box 编号,我们称之为 Parent Box.

    \item \lstinline|int nGhost_|

    ghost cell 层数
  \end{enumerate}
  
  \item 构造函数:
  \begin{enumerate}
    \item \lstinline|AMRMeshHierarchy()|
    \item \lstinline|AMRMeshHierarchy(const ProblemDomain<Dom> &baseDomain,|

          \lstinline|                    const DisjointBoxLayout<Dim> &baseMesh)|

          构造一个仅包含基层网格的 AMR 网格管理器
    \item \lstinline|AMRMeshHierarchy(const std::vector<ProblemDomain<Dim>> &domains,|

          \lstinline|                    const std::vector<DisjointBoxLayout<Dim>> &meshes,|

          \lstinline|                    const std::vector<unsigned>& refRatios)|

          构造一个由用户指定所有网格层的 AMR 网格管理器
  \end{enumerate}
  \item 成员函数:
  \begin{enumerate}
    \item \lstinline|unsigned size() const|

    public 成员函数,返回 AMR 网格的总层数。
    \item \lstinline|const DisjointBoxLayout<Dim>& getMesh(unsigned level) const|

    public 成员函数,返回第 level 层网格。
    \item \lstinline|const ProblemDomain<Dim>& getDomain(unsigned level) const|

    public 成员函数,返回第 level 层网格的 ProblemDomain。

    \item \lstinline|cosnt LevelGhostBoxes<Dim>& getLevelGhostBoxes(unsigned level) const|

    public 成员函数,返回第 level 层网格的界面信息。

    \item \lstinline|unsigned getRefRatioToNextLevel(unsigned level) const|

    public 成员函数,返回第 level 层网格加密到第 level+1 层网格的加密率。

    \item \lstinline|unsigned getParent(unsigned level, unsigned box_id) const|

    public 成员函数,返回第 level 层网格第 box\_id 个 box 的 parent box 编号。

    \item \lstinline|void resetLevel(unsigned level, const DisjointBoxLayout<Dim>& newMesh)|

    public 成员函数,重新设置第 level 层网格,重新生成所需的界面信息、
    Father Box 信息,并做合法性检验。

    \item \lstinline|void push_back(const DisjointBoxLayout<Dim>& newFineMesh, int refRatio = 2)|

    public 成员函数,添加一层细网格,并生成所需信息,做合法性检验。
  \end{enumerate}

  \item 工厂模式:
  \begin{enumerate}
    \item \lstinline|LevelData<T,Dim> createLevelData(|
    
    \lstinline|    unsigned level, |
    
    \lstinline|    int centering [或 const std::vector<int> &centering], |
    
    \lstinline|    int nComps = 1)|

    模板:\lstinline|template<typename T>|

    创建一个第 $level$ 层网格上的 \lstinline|LevelData| 对象

    \item \lstinline|std::vector<LevelData<T,Dim>> createAMRData(|
    
    \lstinline|    int centering [或 const std::vector<int> &centering], |
    
    \lstinline|    int nComps = 1)|

    模板:\lstinline|template<typename T>|

    创建一个整个 AMR 上的数据

    \item \lstinline|std::vector<LevelData<T,Dim>> createAMRDataFaceVector()|

    模板:\lstinline|template<typename T>|

    创建一个整个 AMR 上的 Face-Center 向量,包含 Dim 个分量,第 $i$ 个分量为 Face$i$-Center

    \item \lstinline|LevelOp<Dim> createLevelOp(unsigned level)|

    创建一个第 $level$ 层网格上的 \lstinline|LevelOp| 对象

    \item \lstinline|FuncFiller<Dim> createFuncFiller(unsigned level)|

    创建一个第 $level$ 层网格上的 \lstinline|FuncFiller| 对象

    \item \lstinline|GhostFiller<Dim> createGhostFiller(unsigned level)|

    创建一个第 $level$ 层网格上的 \lstinline|GhostFiller| 对象
  \end{enumerate}
\end{itemize}

\section{有限体积离散}

\subsection{class GhostFiller}

\begin{itemize}
  \item 功能:根据表达式填充某一层网格 \textbf{物理边界} 处的 ghost cell 数据。
  \item 模板:\lstinline|template<int Dim>|
  \item 成员变量:
  \begin{enumerate}
    \item \lstinline|const ProblemDomain<Dim> domain_|
    \item \lstinline|const DisjointBoxLayout<Dim> mesh_|
    \item \lstinline|const LevelGhostBoxes<Dim> ghostBoxes_|
  \end{enumerate}
  \item 构造函数:
  \begin{enumerate}
    \item \lstinline|GhostFiller(const ProblemDomain<Dim>& domain|
    
          \lstinline|              const DisjointBoxLayout<Dim>& mesh,|
    
          \lstinline|              const LevelGhostBoxes<Dim>& ghostBoxes)|

    public 构造函数,根据 AMR 某一层的定义域和计算网格构造一个 \lstinline|GhostFiller| 对象。
  \end{enumerate}
  \item 成员函数:
  \begin{enumerate}
    \item \lstinline|void fillGhosts(LevelData<Real, Dim>& aData, Vector<Array<char,Dim*2>> bcType, unsigned comp = 0) const|

    public 成员函数,以齐次边界条件填充 \lstinline|aData| 第 comp 个分量的所有的 ghost cell,其中
    \lstinline|bcType| 用于描述每个 domain 四条边上的边界条件类型 (D=Dirichlet, N=Neumann, I=Intergrid).
    \item \lstinline|void fillGhosts(LevelData<Real, Dim>&aData, const Vector<Array<Tensor<Real,Dim-1>, Dim*2>>& cDatas, std::vector<std::array<char,Dim*2>> bcType, unsigned comp = 0)|

    public 成员函数,以 \lstinline|cData| 为边界条件填充 \lstinline|aData| 第 comp 个分量的所有的 ghost cell. 
    \lstinline|cData| 的第一个下标表示 Domain 编号, 第二个下标表示边界编号(左右下上).
    % \item \lstinline|void fillGhosts(LevelData<Tensor<Real, Dim>,Dim>& aData, const LevelData<Tensor<Real, Dim>, Dim>* cData) const|

    \item \lstinline|void fillCorners(LevelData<Real, Dim>& aData) const|

    public 成员函数,用外插法填充角上的 ghost

    \item \lstinline|void fillGhostsCFHomo(LevelData<Real, Dim>& aData, unsigned comp = 0, bool useZeroGhost = false) const|

    public 成员函数,齐次粗细插值,填充粗细边界上细网格的 ghost,其中假设粗网格上的值都为 $0$,若设 \lstinline|useZeroGhost=true|,则直接让这些 ghost 等于 $0$.

    % public 成员函数,用 cData 填充所有面的 ghost cell。cData 数组应该包含两个元素,
    % 表示不同 centering 的物理界面的数据。
    % \item \lstinline|void fillGhosts(LevelData<Tensor<Real, Dim>,Dim> &aData) const|
    
    % public 成员函数,以齐次边界条件填充所有面的 ghost cell。
    % \item \lstinline|void doFillGhosts(DataIterator<Tensor<Real, Dim>, Dim> ditr, int D, int side, char type, const T &cData) const;|

    % protected 成员函数,实际进行填充。
  \end{enumerate}
\end{itemize}

% \subsection{class AMRGhostFiller}

% \begin{itemize}
%   \item 功能:根据表达式填充整个 AMR 网格 \textbf{物理边界} 处的 ghost cell 数据。
%   \item 模板:\lstinline|template<int Dim>|
%   \item 成员变量:
%   \begin{enumerate}
%     \item \lstinline|std::vector<GhostFiller> lgf_|

%     private 成员变量,每一层的 GhostFiller。
%   \end{enumerate}
%   \item 构造函数:
%   \begin{enumerate}
%     \item \lstinline|AMRGhostFiller(const std::vector<SplitMultiBoxRectDomain<Dim>> & rds)|

%     public 构造函数,根据 AMR 每一层的计算区域构造一个 \lstinline|AMRGhostFiller| 对象。
%   \end{enumerate}
%   \item 成员函数:
%   \begin{enumerate}
%     \item \lstinline|void fillGhosts(AMRData<Real, Dim>& aData, int D, int side, char type, int) const|

%     public 成员函数,以齐次边界条件填充\lstinline|aData|的一个面的 ghost cell。
%     \item \lstinline|void fillGhosts(AMRData<Real, Dim>, Dim>&aData, int D, int side, char type, const Vector<Vector<Tensor<Real, Dim-1>>>& cData)|

%     public 成员函数,以\lstinline|cData|为边界条件填充\lstinline|aData|的一个面的 ghost cell,其中\lstinline|cData|里的每个 Tensor 对应\lstinline|aData|中每一层、每个Tensor的边界。
    
%     \item \lstinline|void fillGhosts(AMRData<Real,Dim>& ld, const char* types, const Vector<Vector<Tensor<Real,Dim-1>>>* cData) const|

%     public 成员函数,用 cData 填充所有面的 ghost cell。
%     \item \lstinline|void fillGhosts(AMRData<Real, Dim> &ld, const char *types,int) const|

%     public 成员函数,以齐次边界条件填充所有面的 ghost cell。
%   \end{enumerate}
% \end{itemize}

\subsection{class FuncFiller}

\begin{itemize}
  \item 功能:根据表达式填充规则区域 AMR 网格某一层的数据, 或填充物理边界条件。
  \item 模板:\lstinline|template<int Dim>|。
  \item 成员变量:
  \begin{enumerate}
    \item \lstinline|const ProblemDomain<Dim> domain_|
    \item \lstinline|const DisjointBoxLayout<Dim> mesh_|
  \end{enumerate}
  \item 构造函数:
  \begin{enumerate}
    \item \lstinline|GhostFiller(const ProblemDomain<Dim>& domain|
    
          \lstinline|              const DisjointBoxLayout<Dim>& mesh)|

    public 构造函数,根据 AMR 某一层的定义域和计算网格构造一个 \lstinline|FuncFiller| 对象。
  \end{enumerate}
  \item 成员函数:
  \begin{enumerate}
    \item \lstinline|void fillAvr(LevelData<Real, Dim> &aData, const TFunc &expr, bool fillGhosts = true, unsigned comp = 0) const|

    模板:\lstinline|template<typename TFunc>|

    public 成员函数,根据表达式填充 \lstinline|aData| 的第 comp 个分量。
    % \item \lstinline|void fillBdryAvr(LevelData<Tensor<Real, Dim>> &cData, int side, const TFunc &expr, bool fillGhosts = true) const|

    % 模板:\lstinline|template<typename TFunc>|

    % public 成员函数,根据表达式填充 \lstinline|cData| 的一个边界。

    \item \lstinline|void fillBdryAvr(Vector<Array<Tensor<Real,Dim-1>, Dim*2>>& cDatas, const TFunc &expr, int domainBoxID, int D, int centering) const|
    
    模板:\lstinline|template<typename TFunc>|

    public 成员函数,根据表达式填充 \lstinline|cData| 在第 domainBoxID 个 domain 的第 D 个面上的边界条件,该边界条件类型具有给定的 centering。\lstinline|cData| 的第一个下标表示 Domain 编号, 第二个下标表示边界编号(左右下上).

    \item \lstinline|void fillBdryAvr(Vector<Array<Tensor<Real,Dim-1>, Dim*2>>& cDatas, const TFunc &expr, int dataCentering) const|
    
    模板:\lstinline|template<typename TFunc>|

    public 成员函数,根据同一个表达式填充 \lstinline|cData| 的所有信息,\lstinline|dataCentering| 是指对应的 \lstinline|LevelData| 的 centering 类型,其内部会自动匹配边界条件的 centering

    % \item \lstinline|void fillNormalGrad(LevelData<Tensor<Real, Dim>> &cData, int side, const TFunc &expr, bool fillGhosts = true) const|

    % 模板:\lstinline|template<typename TFunc>|

    % public 成员函数,根据表达式填充 \lstinline|cData| 的一个Neumann边界。

    % \item \lstinline|void fillNormalGrad(LevelData<Tensor<Real, Dim>> *cData, const TFunc &expr, bool fillGhosts = true) const|

    % 模板:\lstinline|template<typename TFunc>|

    % public 成员函数,根据表达式填充 \lstinline|cData| 的所有Neumann边界。
  \end{enumerate}
\end{itemize}

\subsection{class LevelOp}

\begin{itemize}
  \item 功能:有限体积法的标准离散格式。
  \item 模板:\lstinline|template<int Dim>|。
  \item 别名:
  \begin{enumerate}
      \item \lstinline|using LD = LevelData<Real, Dim>|
  \end{enumerate}
  \item 成员变量:
  \begin{enumerate}
    \item \lstinline|const ProblemDomain<Dim> domain_|
  \end{enumerate}
  \item 构造函数:
  \begin{enumerate}
    \item \lstinline|LevelOp(const ProblemDomain<Dim>& domain,|
    
          \lstinline|         Communicator comm = MPI_COMM_WORLD)|

    public 构造函数,根据 AMR 某一层的定义域构造一个 \lstinline|LevelOp| 对象。
  \end{enumerate}
  \item 成员函数:
  \begin{enumerate}  
    \item \lstinline|void computeLaplacian(const LD &srcData, LD &dstData) const|

    public 成员函数,标准四阶 Laplace 算子
    \item \lstinline|void computeHelmoltz(const LD &srcData, LD &dstData, Real alpha, Real beta) const|

    public 成员函数,标准四阶 Helmoltz 算子 $\alpha I + \beta \Delta$
    \item \lstinline|void computeGradient(const LD &srcData, LD &dstData) const|

    public 成员函数,标准四阶梯度算子,其中 srcData 必须是单分量且 Cell-Center,dstData 必须是 Dim 分量且 Cell-Center 或 Face-Center,若为 Face-Center,则第 $i$ 个分量必须是 Face$i$-Center
    \item \lstinline|void computeDivergence(const LD &srcData, LD &dstData) const|

    public 成员函数,标准四阶散度算子,其中 srcData 必须具有 Dim 个分量且第 $i$ 个分量必须是 Face$i$-Center,而 dstData 必须是单分量且 Cell-Center
    \item \lstinline|void computeCurl(const LD &srcData, LD &dstData) const|

    public 成员函数,标准四阶散度算子,其中 srcData 必须具有 Dim 个分量且第 $i$ 个分量必须是 Face$i$-Center,二维情形下, dstData 必须是单分量且 Node-Center
    \item \lstinline|void computeMagnitude(const LD &srcData, LD &dstData) const|

    public 成员函数,计算向量的大小,其中 srcData 必须具有 Dim 个分量且 Cell-Center,而 dstData 必须是单分量且 Cell-Center
    \item \lstinline|void computeD2(const LD &srcData, LD &dstData) const|

    public 成员函数,标准四阶 $\frac{\partial^2}{\partial x_d^2}$ 算子,srcData 必须是单分量,dstData 必须是 Dim 分量且与 srcData 的 centering 一致,它的第 $d$ 个分量保存 $\frac{\partial^2}{\partial x_d^2}$ 的结果
    \item \lstinline|void filterFace2Cell(const LD &srcData, LD &dstData) const|

    public 成员函数,将 Face-Center 的数据保四阶精度地转换为 Cell-Center 的数据。其中 srcData 是 Face-Center 的,
    dstData 必须和 srcData 具有一样的分量个数且都是 Cell-Center 的
    \item \lstinline|void filterCell2Face(const LD &srcData, LD &dstData) const|

    public 成员函数,将 Cell-Center 的数据保四阶精度地转换为 Face-Center 的数据。其中 srcData 是单分量且 Cell-Center 的,dstData 必须为 Dim 分量且第 $i$ 个分量必须是 Face$i$-Center
    \item \lstinline|void relaxJacobi(const LD &src, const LD &res, LD &dst, Real alpha = 0, Real beta = 1, Real jW = 0.5) const|

    public 成员函数,针对 Helmoltz 方程 $\alpha \phi + \beta \Delta \phi = r$ 的标准四阶加权 Jacobi 迭代,权重为 jW
    \item \lstinline|std::vector<Real> computeNorm(const LD &aData, int q) const|

    public 成员函数,计算 LevelData 的 $L_q$ 误差范数,q只能设为 $0,1,2$,每个分量分别计算,返回值的 size 是分量个数
  \end{enumerate}
\end{itemize}

\subsection{class AMRIntergridOp}

\begin{itemize}
  \item 功能:提供 AMR 数据层的一些操作,包括同层交换、粗细层间限制与插值。
  \item 模板:\lstinline|template<int Dim>|。
  \item 别名:
  \begin{enumerate}
    \item \lstinline|using LD = LevelData<Real, Dim>|
  \end{enumerate}
  \item 成员变量:
  \begin{enumerate}
    \item \lstinline|const AMRMeshHierarchy<Dim>& amrHier_|
    \item \lstinline|std::vector<LevelOp<Dim>>& lvOp_|
 \end{enumerate}
  \item 构造函数:
  \begin{enumerate}
    \item \lstinline|AMRIntergridOp(const AMRMeshHierarchy<Dim>& amrHier)|
  \end{enumerate}
  \item 成员函数:
  \begin{enumerate}
    \item \lstinline|void exchangeInLevel(std::vector<LD> &aData) const|

    public 成员函数,对 \lstinline|aData| 进行同层数据交换,并填充至相应的 Ghost 中。

    \item \lstinline|void averageToCoarse(std::vector<LD> &aData) const|

    public 成员函数,将除基网格外的每一层数据取平均传递给对应的粗网格。

    \item \lstinline|void averageToCoarse(const LD& fineData, |
    
          \lstinline|                         LD& coarseData, unsigned fineLevel) const|
    
    public 成员函数,同上,但只做第 fineLevel 层。

    \item \lstinline|void interpolateToFineGhost(const LD& coarseData,|
    
          \lstinline|                                 LD& fineData, unsigned fineLevel) const|

    public 成员函数,同上,通过四阶粗细插值将第 fineLevel - 1 层的数据插值到第 fineLevel 层的 Ghost。

    \item \lstinline|void constantInterpolateIncr(const LD& coarseData, |
    
          \lstinline|                                  LD& fineData, unsigned fineLevel) const|

    public 成员函数,将粗网格的非 Ghost 数据分片常数地插值到细网格,并加给 \lstinline|fineData|
        \item \lstinline|void autoInterpolateIncr(const LD& coarseData, |
    
          \lstinline|                                  LD& fineData, unsigned fineLevel, int order) const|

    public 成员函数,将粗网格的非 Ghost 数据分片高阶地插值到细网格,并加给 \lstinline|fineData|,其中 order 只能设为 $1,2,4$,内部采用自动对称化插值格式
    \item \lstinline|Vector<Real> computeNorm(const Vector<LD> &aDatas, int q) const|

    public 成员函数,计算整个 AMR 数据的 $L_q$ 误差范数,q只能设为 $0,1,2$,每个分量分别计算,返回值的 size 是分量个数
  \end{enumerate}
\end{itemize}



% \subsection{class AMRFuncFiller}

% \begin{itemize}
%   \item 功能:根据表达式填充规则区域 AMR 网格数据。
%   \item 模板:\lstinline|template<int Dim>|。
%   \item 成员变量:
%   \begin{enumerate}
%     \item \lstinline|std::vector<FuncFiller> funcfillers_|

%     private 成员变量,每一层的 FuncFiller
%   \end{enumerate}
%   \item 构造函数:
%   \begin{enumerate}
%     \item \lstinline|AMRFuncFiller(const std::vector<SplitMultiRectDomain<Dim>>& rds)|
%   \end{enumerate}
%   \item 成员函数:
%   \begin{enumerate}
%     \item \lstinline|void fillAvr(AMRData<Real, Dim> &aData, int centering, const TFunc &expr, bool fillGhosts = true) const|

%     模板:\lstinline|template<typename TFunc>|

%     public 成员函数,根据表达式填充 \lstinline|aData|。
%     \item \lstinline|void fillBdryAvr(Vector<Vector<Tensor<Real, Dim - 1>>> &aData, int centering, int D, int side, const TFunc &expr, bool fillGhosts = true) const|

%     模板:\lstinline|template<typename TFunc>|

%     public 成员函数,根据表达式填充 \lstinline|aData| 的一个边界,每一层都要填,\lstinline|aData| 的第一个下标
%     表示层号,第二个下标表示 Box 编号,下同。

%     \item \lstinline|void fillBdryAvr(Vector<Vector<Tensor<Real, Dim - 1>>> *aData, int centering, const TFunc &expr, bool fillGhosts = true) const|

%     模板:\lstinline|template<typename TFunc>|

%     public 成员函数,根据表达式填充 \lstinline|aData| 的所有边界,每一层都要填,\lstinline|aData[0~2^Dim-1]|表示所有规则边界的边界值。

%     \item \lstinline|void fillNormalGrad(Vector<Vector<Tensor<Real, Dim - 1>>> &aData, int centering, int D, int side, const TFunc &expr, bool fillGhosts = true) const|

%     模板:\lstinline|template<typename TFunc>|

%     public 成员函数,根据表达式填充 \lstinline|aData| 的一个Neumann边界,每一层都要填,

%     \item \lstinline|void fillNormalGrad(Vector<Vector<Tensor<Real, Dim - 1>>> *aData, int centering, const TFunc &expr, bool fillGhosts = true) const|

%     模板:\lstinline|template<typename TFunc>|

%     public 成员函数,根据表达式填充 \lstinline|aData| 的所有Neumann边界,每一层都要填,\lstinline|aData[0~2^Dim-1]|表示所有规则边界的边界值。
%   \end{enumerate}
% \end{itemize}


% \subsection{class EmFuncFiller: public FuncFiller<Dim>}

% \begin{itemize}
%   \item 功能:根据表达式填充不规则网格数据。
%   \item 模板:\lstinline|template<int Dim>|
%   \item 成员变量:
%   \begin{enumerate}
%     \item \lstinline|const SplitMultiBoxEmbeddedGrid<Dim>& emGrid_|

%     private 成员变量,表示不规则网格定义域。
%   \end{enumerate}

  
  % \item 构造函数:
  % \begin{enumerate}
  %   \item \lstinline|EmFuncFiller(const MultiBoxEmbeddedGrid<Dim>& emGrid)|
  % \end{enumerate}
  % \item 成员函数:
  % \begin{enumerate}
  %   \item \lstinline|void fillAvr(LevelData<Tensor<Real, Dim>,Dim> &aData, int centering, const TFunc &f, bool fillGhosts = true) const|

  %   模板:\lstinline|template<typename TFunc>|

  %   public 成员函数,根据表达式填充 \lstinline|aData|。
  %   \item \lstinline|void fillBdryAvr(std::vector<Tensor<Real, Dim-1>> &aData, const char *, const TFunc &f) const|

  %   模板:\lstinline|template<typename TFunc>|

  %   public 成员函数,根据表达式填充 \lstinline|aData| 的不规则边界值。

  %   \item \lstinline|void fillNormalGrad(std::vector<Tensor<Real, Dim-1>> &aData, const char *, const TFunc &f) const|

  %   模板:\lstinline|template<typename TFunc>|

  %   public 成员函数,根据表达式填充 \lstinline|aData| 的不规则Neumann边界值。
  % \end{enumerate}
% \end{itemize}

% \begin{note}
%   不规则区域暂时不管
% \end{note}

% \subsection{class OutrangeData}

% \begin{itemize}
%   \item 功能:用于处理 PLG 离散模板中超出2层Ghost Cell、Slave 超出2层Ghost Cell的情况。只能在同层数据中使用。
%   \item 成员变量:
%   \begin{enumerate}
%     \item \lstinline|Map<int, Set<iVec>> idx2Send_|

%     private 成员变量,表示 (neighbor proc id -> cell index to send)。
%     \item \lstinline|Map<int, Set<iVec>> idx2Recv_|

%     private 成员变量,表示 (neighbor proc id -> cell index to receive)。
%     \item \lstinline|Map<int, Vector<char>> sendBuf_|

%     private 成员变量,表示 (neighbor proc id -> send buffer),其中 send buffer 是将数据打包后形成的char缓冲区。
%     \item \lstinline|Map<int, Vector<char>> recvBuf_|

%     private 成员变量,表示 (neighbor proc id -> recv buffer),其中 recv buffer 是将数据打包后形成的char缓冲区。
%   \end{enumerate}
%   \item 构造函数:
%   \begin{enumerate}
%     \item \lstinline|OutrangeData(Map<int, Set<iVec>> map, bool init_by_recv = true)|

%     构造函数,\lstinline|init_by_recv = true|表示已知\lstinline|idx2Recv_|,并用\lstinline|idx2Recv|初始化\lstinline|idx2Send_|。若 \lstinline|init_by_recv = false|则为使用\lstinline|idx2Recv_|初始化\lstinline|idx2Send_|
%   \end{enumerate}
%   \item 成员函数:
%   \begin{enumerate}
%     \item \lstinline|void linearIn(TF handler, const Map<int, T>& in)|

%     模板:\lstinline|template<class TF, class T>|

%     public 成员函数,通过\lstinline|handler|处理将\lstinline|in|打包到\lstinline|sendBuf_|中。
%     \item \lstinline|void linearOut(TF handler, Map<int, T>& out)|

%     模板:\lstinline|template<class TF, class T>|
    
%     public 成员函数,用于将数据打包到\lstinline|out|中。
%     \item \lstinline|void syncOutrangeInfo(bool init_by_recv = true)|

%     private 成员函数,若 \lstinline|init_by_recv = true| 则根据 \lstinline|idx2Recv| 设置 \lstinline|idx2Send|,若 \lstinline|init_by_recv = false|则反之。
%     \item \lstinline|void exchangeData()|

%     public 成员函数,发送\lstinline|sendBuf_|并接收\lstinline|recvBuf_|
%   \end{enumerate}
% \end{itemize}


\section{多重网格求解器}

\subsection{class LinearSolver}

\begin{itemize}
  \item 功能:线性方程的求解器基类,其派生类可以是直接法也可以是迭代法
  \item 模板:\lstinline|template<int T>|
  % \item 成员变量:
  % \begin{enumerate}
  %   \item \lstinline|const SplitMultiBoxRectDomain<Dim> grid_|

  %   protected 成员变量,表示目标规则定义域。
  % \end{enumerate}
  % \item 构造函数:
  % \begin{enumerate}
  %   \item \lstinline|LevelOp(const SplitMultiBoxRectDomain<Dim>& grid)|
  % \end{enumerate}
  \item 成员函数:
  \begin{enumerate}
    \item \lstinline|virtual void apply(T& dst, const T& src) const = 0|

    public 成员函数,计算 \lstinline|dst = L(src)|.
  
    \item \lstinline|virtual void computeResidual(T& dst, const T& src, const T &rhs) const|

    public 成员函数,计算 \lstinline|dst = rhs - L(src)|.

    \item \lstinline|virtual Vector<double> computeNorm(const LD& aData, int q) const|

    public 成员函数,计算每个分量的 $L^q$ 范数。

    \item \lstinline|virtual void relax(const T& phi, const T &rhs, T& smoothed) = 0|
    
    public 成员函数,对方程 \lstinline|L(phi) = rhs| 进行一次松弛(例如 Jacobi 迭代)。

    \item \lstinline|virtual void solve(const T& phi, const T &rhs) = 0|
    
    public 成员函数,对方程 \lstinline|L(phi) = rhs| 进行求解(可以是多次迭代求解,也可以是直接求解)。
  \end{enumerate}
\end{itemize}

% \subsection{class MGLevelOp : public LevelOp<T>}

% \begin{itemize}
%   \item 功能:多重网格某层的算子基类。
%   \item 模板:\lstinline|template<int T>|
%   % \item 成员变量:
%   % \begin{enumerate}
%   %   \item \lstinline|const SplitMultiBoxRectDomain<Dim> grid_|

%   %   protected 成员变量,表示目标规则定义域。
%   % \end{enumerate}
%   % \item 构造函数:
%   % \begin{enumerate}
%   %   \item \lstinline|LevelOp(const SplitMultiBoxRectDomain<Dim>& grid)|
%   % \end{enumerate}
%   \item 成员函数:
%   \begin{enumerate}
%     \item \lstinline|virtual void applySmooth(const T& phi, const T &rhs, T& smoothed) const = 0|
%     public 成员函数,虚函数,对方程 \lstinline|L(phi) = rhs| 进行一次某种松弛(例如加权 Jacobi)。
%     \begin{note}
%         作泛化用
%     \end{note}

%     \item \lstinline|virtual void bottomSolve(T &phi, const T& rhs) = 0|

%     public 成员函数,虚函数,求解方程 \lstinline|L(phi) = rhs|。
%     \begin{note}
%         作泛化用
%     \end{note}
%   \end{enumerate}
% \end{itemize}

% \subsection{class SpatialOperator}

% \begin{itemize}
%   \item 功能:作为不规则空间离散算子的基类(省略了部分实现时要用到的辅助函数)。
%   \item 模板:\lstinline|template<int Dim>|
%   \item 成员变量:
%   \begin{enumerate}
%     \item \lstinline|const SplitMultiBoxEmbeddedGrid<Dim>& emGrid_|

%     protected 成员变量,表示目标不规则网格。
%   \end{enumerate}
  % \item 构造函数:
  % \begin{enumerate}
  %   \item \lstinline|SpatialOperator(const MultiBoxEmbeddedGrid<Dim>& em_grid)|
  % \end{enumerate}
  % \item 成员函数:
  % \begin{enumerate}
  %   \item \lstinline|void buildSampleMatrix(int dataIdx, const Stencil& stencil, char bdryType, const rVec& refLoc, Tensor<Real,2>& mout)|

  %   public 成员函数,构建最小二乘法需要的增广采样矩阵。
  % \end{enumerate}
% \end{itemize}

% \begin{note}
%     不规则区域暂时不管
% \end{note}

\subsection{class HelmoltzJacobiSmoother: public LinearSolver< LevelData<Real, Dim> >}

\begin{itemize}
  \item 功能:作用于 Helmoltz 方程 \lstinline|alpha * I + beta * Lap| 的光滑子,内部采用加权 Jacobi 迭代。
  \item 模板:\lstinline|template<int Dim>|
  \item 继承:\lstinline|LinearSolver< LevelData<Real, Dim> >|
  \item 别名:
  \begin{enumerate}
      \item \lstinline|using LD = LevelData<Real, Dim>|
  \end{enumerate}
  \item 成员变量:
  \begin{enumerate}
      \item \lstinline|Real alpha_|
      \item \lstinline|Real beta_|
      \item \lstinline|Real jacobiWeight_|
  \end{enumerate}
  \item 构造函数:
  \begin{enumerate}
      \item \lstinline|HelmoltzJacobiSmoother(Real alpha, Real beta, Real jacobiWeight=0)|
  \end{enumerate}
  \item 成员函数:
  \begin{enumerate}
    \item \lstinline|virtual void apply(LD& dst, const LD& src) const|

    public 成员函数,计算 \lstinline|dst = (alpha * I + beta * Lap) (src)|
    \item \lstinline|virtual void relax(const LD& phi, const LD& rhs, LD& smoothed)|

    public 成员函数,对 Helmoltz 方程进行一次加权 Jacobi 迭代。

    \item \lstinline|virtual void solve(const LD& phi, const LD& rhs)|

    public 成员函数,也是对 Helmoltz 方程进行一次加权 Jacobi 迭代。
  \end{enumerate}
\end{itemize}

% \subsection{class AMRSpatialOperator(初步设计)}

% \begin{itemize}
%   \item 功能:表示不规则区域的 AMR 网格上直接离散的任意算子。
%   \item 模板:\lstinline|template<typename SpOp, int Dim>|
%   \item 成员变量:
%   \begin{enumerate}
%     \item \lstinline|std::vector<SpOp> levelSpOp|

%     private 成员变量,存放每层网格上的算子。
%   \end{enumerate}
%   \item 构造函数:
%   \begin{enumerate}
%     \item \lstinline|AMRSpatialOperator(const std::vector<SplitMultiBoxEmbeddedGrid<Dim>>& em_grid, int centering, BoundaryType = Dirichlet)|
%   \end{enumerate}
%   \item 成员函数:
%   \begin{enumerate}
%     \item \lstinline|void apply(const AMRData<Real,Dim>& u, const T1 & irBcData, const T2 &rhs, Real beta, AMRData<Real,Dim>& dest) const|

%     模板:\lstinline|Template<typename T1, typename T2>|

%     public 成员函数,计算 \lstinline|beta*SpOp(u) + rhs|。
%   \end{enumerate}
% \end{itemize}

% \subsection{class BlockRelaxation: public SpatialOperator<Dim>(初步设计)}

% \begin{itemize}
%   \item 功能:对应多重网格中的块松弛算法,在实现时需要将不规则区域数据传输到同一个进程中(默认0号)。
%   \item 模板:\lstinline|template<int Dim, class _SparseSolver>|
%   \item 别名:
%   \begin{enumerate}
%     \item \lstinline|using iVec = Vec<int, Dim>|
%     \item  \lstinline|using RS = RowSparse<iVec,iVec>|
%   \end{enumerate}
%   \item 成员变量:
%   \begin{enumerate}
%     \item \lstinline|const Vector<RS>& spLap_|

%     private 成员变量,对应不规则边界拉普拉斯算子的线性变换。
%     \item \lstinline|Vector<RS> offDiagBlock_|

%     private 成员变量,对应与不规则边界拉普拉斯算子分块矩阵中同行的分块矩阵。
%     \item \lstinline|int blockSize_|

%     private 成员变量,表示待分解矩阵的大小。
%     \item \lstinline|RowSparse<int, int> diagBlock_|

%     private 成员变量,表示对角分块矩阵。
%     \item \lstinline|_SparseSolver solver_|

%     private 成员变量,表示稀疏求解器。
%     \item \lstinline|MPI_Comm comm_|

%     private 成员变量,表示MPI通讯器
%   \end{enumerate}
%   \item 构造函数:
%   \begin{enumerate}
%     \item \lstinline|BlockRelaxation(const SplitMultiBoxEmbeddedGrid<Dim>& emGrid, const RS& aLapSpOp, Real alpha, Real beta, const _SparseSolver &aSolver, MPI_Comm aComm = MPI_COMM_WORLD)|

%     public 构造函数。
%   \end{enumerate}
%   \item 成员函数:
%   \begin{enumerate}
%     \item \lstinline|void apply(LevelData<Real,2>& phi, const T& irBcData, const LevelData<Real,2>& rhs, int proc_id = 0) const|

%     public 成员函数,对目标函数作用块松弛。
%   \end{enumerate}
% \end{itemize}


% \subsection{class HelmoltzDirectSolver : public LinearSolver< LevelData<Real, Dim> >}

% \begin{itemize}
%   \item 功能:作用于 Helmoltz 方程的直接法求解器。
%   \item 模板:\lstinline|template<int Dim, class _Solver>|
%   \item 继承:\lstinline|LinearSolver< LevelData<Real, Dim> >|
% \item 别名:
%   \begin{enumerate}
%       \item \lstinline|using LD = LevelData<Real, Dim>|
%   \end{enumerate}
%   \item 成员变量:
%   \begin{enumerate}
%     \item \lstinline|const DisjointBoxLayout<Dim>& mesh_|
%     \item \lstinline|const LevelGhostBoxes<Dim>& ghostBoxes_|
%     \item \lstinline|const ProblemDomain<Dim>& domain_|
%     \item \lstinline|int centering_|
%     \item \lstinline|Real alpha_, beta_|
%     \item \lstinline|bool pureNeumann_|
%     \item \lstinline|int numUnKnown_|
%     \item \lstinline|int numActualUnkowns_|
%     \item \lstinline|LevelData<int,Dim> unknownMap_|
%     \item \lstinline|_Solver solver_|
%     \item \lstinline|Map<iVec, int> irVec2Idx_|

%     private 成员变量,用于将RowSparse中行标转换为Solver矩阵中的下标。
%     \item \lstinline|MPI_Comm comm_|
%   \end{enumerate}
%   \item 构造函数:
%   \begin{enumerate}
%     \item \lstinline|HelmoltzDirectSolver(const DisjointBoxLayout<Dim>& mesh, |
    
%           \lstinline|               const LevelGhostBoxes<Dim>& ghostBoxes,|

%           \lstinline|               const ProblemDomain<Dim>& domain,|
          
%           \lstinline|               int centering, |
          
%           \lstinline|               Real alpha, |
          
%           \lstinline|               Real beta, |
          
%           \lstinline|               const _Solver &aSolver, |
          
%           \lstinline|               MPI_Comm aComm = MPI_COMM_WORLD)|

%     构造一个求解方程 \lstinline|(alpha * I + beta * Lap) phi = rhs| 的直接法求解器
%   \end{enumerate}
%   \item 成员函数:
%   \begin{enumerate}
%     \item \lstinline|virtual void solve(LD &phi, const LD& rhs)|

%     直接法求解 Helmoltz 方程
%   \end{enumerate}

%   \item 禁用函数:
%     \begin{enumerate}
%         \item \lstinline|virtual void apply(LD& dst, const LD& src) const|
%         \item \lstinline|virtual void relax(const LD& phi, const LD& rhs, LD& smoothed)|
%     \end{enumerate}
% \end{itemize}

\subsection{class MGIntergrid}

\begin{itemize}
  \item 功能:相邻的多重网格隐含层之间用于数据交互的类。
  \item 模板:\lstinline|template<int Dim>|
  \item 别名:
  \begin{enumerate}
      \item \lstinline|using LD = LevelData<Real, Dim>|
  \end{enumerate}
  \item 成员变量:暂无,以后可能会有两个 EmGrid.
  \item 成员函数:
  \begin{enumerate}
    \item \lstinline|void applyRestrict(const LD& fineData, LD& coarseData) const|

    将细网格非 Ghost 的值取平均传递给粗网格

    \item \lstinline|void applyInterpolate(const LD& coarseData, LD& fineData) const|

    将粗网格非 Ghost 的值分片常数地插值给细网格。

    % \item \lstinline|void linearInterpolate(const LD& coarseData, LD& fineData) const|

    % 将粗网格非 Ghost 的值线性插值给细网格。
    % \begin{note}
    %     线性插值可以暂时不实现
    % \end{note}
    \end{enumerate}

\end{itemize}

\subsection{class AMRMGLevelOp : public LinearSolver< LevelData<Real, Dim> >}

\begin{itemize}
  \item 功能:AMR多重网格算法中某一层的算子,主要功能是生成 AMR 层间的多重网格隐含层,
             并将隐含层中的小 V-Cycle 封装成一个 relax,
             与 ChomboDesign 中 Figure 4.1 所述算法对应。
  \item 模板:\lstinline|template<int Dim>|
  \item 别名:
  \begin{enumerate}
      \item \lstinline|using LD = LevelData<Real, Dim>|
  \end{enumerate}
  \item 成员变量: 略
  % \begin{enumerate}
  %     \item \lstinline|bool defined_|
  %     \item \lstinline|unsigned nLevel_|
  %     \item \lstinline|std::vector<DisjointBoxLayout<Dim>> meshs_|

  %     这里的网格数组不是用户的 AMR 网格,而是某一层 AMR 网格和它底下的
  %     多重网格隐含层(MG implicit levels)。
      
  %     例如对于中间层的 AMR 网格,有 $\log(r)$ 个隐含层($r$ 是这层 AMR 网格的加密率)

  %     对于底层 AMR 网格,有 $k$ 个隐含层($k$ 是将底层网格以 $2$ 的加密率进行加粗,直到无法继续加粗为止,所生成的网格总层数)
  %     \item \lstinline|std::vector< std::unique_ptr< LinearSolver<LD> > > smoothers_|

  %     对于 Helmoltz 方程而言,这里的 \lstinline|smoothers_| 是 \lstinline|HelmoltzJacobiSmoother|
  %     \item \lstinline|std::vector< std::unique_ptr< MGIntergrid<Dim> > > mgIntergridOps_|
  %     \item \lstinline|std::shared_ptr< LinearSolver<LD> > bottomSolver_|

  %     对于 Helmoltz 方程而言,这里的 \lstinline|bottomSolver_| 也是 \lstinline|HelmoltzJacobiSmoother|。
  %     不过我们也留下一个泛化接口,以便以后加入某种底层求解器。
  %     \item \lstinline|std::vector<FuncFiller<Dim>> fcFillers_|
      
  %     每个多重网格隐含层的函数填充器
  %     \item \lstinline|std::vector<GhostFiller<Dim>> gstFillers_|
      
  %     每个多重网格隐含层的物理 ghost 填充器
  %     \item \lstinline|int nComps_|

  %     变量的分量个数
  %     \item \lstinline|std::vector<int> centering|

  %     每个分量的 centering 类型
  %     \item \lstinline|std::vector<LD> levelRes_|

  %     每一个多重网格隐含层的残差
  %     \item \lstinline|std::vector<LD> levelCorr_|

  %     每一个多重网格隐含层的校正量
  %     \item \lstinline|std::vector<LD> levelAux_|

  %     每一个多重网格隐含层的辅助变量,用于数据倒腾来倒腾去。
  %     提前创建好所有必要的 LevelData,可以避免在计算时反复开辟或释放内存导致的巨大耗时。
  %     \item \lstinline|int preSmooth_|
  %     \item \lstinline|int postSmooth_|
  %     \item \lstinline|int bottomSmooth_|
      
  %     这个参数表示最底层的迭代次数,当底层求解器为空时,建议用几十次底层迭代来替代底层求解。但是对于 AMR 相邻层间的多重网格隐含层,\lstinline|bottomSmooth_| 应当设置为0

  %     \item \lstinline|bool useFMG_|

  %     是否使用 FMG-Cycle
  %     \item \lstinline|Vector<Array<char, 2 * Dim>> bcType_|
      
  %     物理边界的条件类型
  % \end{enumerate}
  \item 构造函数:
  \begin{enumerate}
    \item \lstinline|AMRMGLevelOp(const ProblemDomain<Dim>& domain_)|
  \end{enumerate}
  \item 成员函数:
  \begin{enumerate}    
    \item \lstinline|void define(const DisjointBoxLayout<Dim>& finestMesh, |
    
    \lstinline|              Real alpha, |
    
    \lstinline|              Real beta, |

    \lstinline|              int nComps, |

    \lstinline|              const std::vector<int>& centering,|

    
    \lstinline|              unsigned nLevel, |
    
    \lstinline|              LevelOp<LD> *bottomSolver = nullptr)|

    public,最细网格、定义 Helmoltz 方程的系数、多重网格层数、底层求解器。调用 \lstinline|getCoarsened(2)|
    来生成需要的所有粗网格.

    \item \lstinline|void setBCType(const Vector<Array<char, 2 * Dim>> &bcType)|

    public,设置物理边界条件类型
    
    \item \lstinline|void setParam(int preSmooth,int postSmooth,int bottomSmooth)|

    public,设置多重网格算法的参数

    \item \lstinline|void switchFMG(bool useFMG)|

    public, 设置是否使用 FMG-Cycle,方便用户随时切换 V-Cycle 和 FMG-Cycle

    \item \lstinline|void relax(const LD& phi, const LD& rhs, LD& smoothed)|

    public,调用用户指定的 Cycle 进行一次多重网格松弛

    \item \lstinline|void apply(LD& dst, const LD& src) const|

    public,在最细一层作用 Helmoltz 算子
    \end{enumerate}

\end{itemize}

\subsection{class AMRMultigrid}

\begin{itemize}
  \item 功能:AMR 多重网格算法。注意,每一个实例必须要明确整个 AMR 网格以及变量的分量个数、centering类型。对于不同的变量(比如速度、压强),不能使用同一个 AMRMultigrid 实例。
  \item 模板:\lstinline|template<int Dim>|
  \item 别名:
  \begin{enumerate}
      \item \lstinline|using LD = LevelData<Real, Dim>|
  \end{enumerate}
  \item 成员变量: 略
  % \begin{enumerate}
  %     \item \lstinline|bool defined_|
  %     \item \lstinline|const AMRMeshHierarchy<Dim>& amrHier_|
      
  %     AMR网格管理器
  %     \item \lstinline|std::unique_ptr<AMRIntergridOp<Dim>> amrIntergridOp_|
      
  %     负责粗细网格之间插值的类
  %     \item \lstinline|std::vector<FuncFiller<Dim>> fcFillers_|
      
  %     每层 AMR 网格上的函数填充器
  %     \item \lstinline|std::vector<GhostFiller<Dim>> gstFillers_|
      
  %     每层 AMR 网格上的物理 ghost 填充器
  %     \item \lstinline|std::vector< std::unique_ptr< AMRMGLevelOp<Dim> > > mgSmoothers_|

  %     它封装了 ChomboDesign Figure 4.1 所示的 mgRelax 算法,
  %     在 AMRMultigrid 类中无需知道其内部的细节。

  %     \item \lstinline|unsigned nComps_|

  %     分量个数
  %     \item \lstinline|std::vector<int> centering_|

  %     每个分量的 centering 类型
  %     \item \lstinline|std::vector<LD> levelRes_|

  %     存储每一层的残差
  %     \item \lstinline|std::vector<LD> levelCorr_|

  %     存储每一层的校正量
  %     \item \lstinline|std::vector<LD> levelAux_|

  %     每一层的辅助变量,用于数据倒腾来倒腾去
  %     \item \lstinline|int maxIter_|
  %     \item \lstinline|Real relTol_|
  %     \item \lstinline|Real stallTol_|
  % \end{enumerate}
  \item 构造函数:
  \begin{enumerate}
    \item \lstinline|AMRMultigrid(const AMRMeshHierarchy<Dim>& amrHier)|

    初始化参数为默认值
  \end{enumerate}
  \item 成员函数:
  \begin{enumerate}
    \item \lstinline|void define(Real alpha,|

          \lstinline|              Real beta,|

          \lstinline|              unsigned nComps,|

          \lstinline|              const std::vector<int>& centering,|

          \lstinline|              int bottomNumLevel,|

          \lstinline|              int jacobiW,|

          \lstinline|              std::shared_ptr<LinearSolver<LD>> bottomSolver = nullptr)|

    public,定义 AMR 网格、 Helmoltz 方程的系数、分量个数、各分量centering类型、底层多重网格层数、底层求解器,内部要初始化 \lstinline|mgSmoothers_|。

    \item \lstinline|void define(Real alpha,|

          \lstinline|              Real beta,|

          \lstinline|              int centering,|

          \lstinline|              int bottomNumLevel,|

          \lstinline|              int jacobiW,|

          \lstinline|              std::shared_ptr<LinearSolver<LD>> bottomSolver = nullptr)|
    
    public,为单分量的变量提供的一个更加方便的 define 接口。
    
    \item \lstinline|void setParam(int preSmooth,|
          
          \lstinline|                int postSmooth,|
          
          \lstinline|                int bottomSmooth,       // 底层迭代次数|
          
          \lstinline|                int maxIter,            // 最大迭代次数|
          
          \lstinline|                Real relTol,            // 可容忍的最大相对误差|
          
          \lstinline|                Real stallTol,          // 可容忍的最小残差下降率|
          
          \lstinline|                bool useFMG = false,  // 是否使用 FMG-Cycle|
          
          \lstinline|                int FMGInterpOd = 1)    // 在 FMG-Cycle 中使用多少阶插值 [0/1/2/4]|

    public,设置多重网格算法的参数。

    \item \lstinline|void solve(|
      
      \lstinline|    Vector<LD> &phi,|
      
      \lstinline|    const Vector<LD> &rhs,|
      
      \lstinline|    const Vector<Vector<Array<Tensor<Real, Dim - 1>, 2 * Dim>>> &cDatas,|
      
      \lstinline|    const Vector<Array<char, 2 * Dim>> &bcType,|
      
      \lstinline|    Wrapper_Silo<Dim> *dbgSiloOut = nullptr)|

    public,用多重网格求解方程,右端项为 rhs, 整个 AMR 的边界条件数据为 cDatas(\lstinline|cDatas[i]| 表示第 $i$ 层的边界条件数据,可以由 \lstinline|FuncFiller| 生成),边界条件类型为 bcType,最后一个参数用于输出中间结果,
    如果传入一个 \lstinline|Wrapper_Silo| 指针,则可以在 Silo 中输出每一步迭代的解与残差。
  \end{enumerate}
\end{itemize}

\end{document}





